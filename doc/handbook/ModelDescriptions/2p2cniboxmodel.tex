%%%%%%%%%%%%%%%%%%%%%%%%%%%%%%%%%%%%%%%%%%%%%%%%%%%%%%%%%%%%%%%%%
% This file has been autogenerated from the LaTeX part of the   %
% doxygen documentation; DO NOT EDIT IT! Change the model's .hh %
% file instead!!                                                %
%%%%%%%%%%%%%%%%%%%%%%%%%%%%%%%%%%%%%%%%%%%%%%%%%%%%%%%%%%%%%%%%%

\-This model implements a non-\/isothermal two-\/phase flow of two compressible and partly miscible fluids $\alpha \in \{ w, n \}$. \-Thus each component $\kappa \{ w, a \}$ can be present in each phase. \-Using the standard multiphase \-Darcy approach a mass balance equation is solved\-: \begin{eqnarray*} && \phi \frac{\partial (\sum_\alpha \varrho_\alpha X_\alpha^\kappa S_\alpha )}{\partial t} - \sum_\alpha \text{div} \left\{ \varrho_\alpha X_\alpha^\kappa \frac{k_{r\alpha}}{\mu_\alpha} \mbox{\bf K} (\text{grad}\, p_\alpha - \varrho_{\alpha} \mbox{\bf g}) \right\}\\ &-& \sum_\alpha \text{div} \left\{{\bf D}_{\alpha, pm}^\kappa \varrho_{\alpha} \text{grad}\, X^\kappa_{\alpha} \right\} - \sum_\alpha q_\alpha^\kappa = 0 \qquad \kappa \in \{w, a\} \, , \alpha \in \{w, n\} \end{eqnarray*} \-For the energy balance, local thermal equilibrium is assumed which results in one energy conservation equation for the porous solid matrix and the fluids\-: \begin{eqnarray*} && \phi \frac{\partial \left( \sum_\alpha \varrho_\alpha u_\alpha S_\alpha \right)}{\partial t} + \left( 1 - \phi \right) \frac{\partial (\varrho_s c_s T)}{\partial t} - \sum_\alpha \text{div} \left\{ \varrho_\alpha h_\alpha \frac{k_{r\alpha}}{\mu_\alpha} \mathbf{K} \left( \text{grad}\, p_\alpha - \varrho_\alpha \mathbf{g} \right) \right\} \\ &-& \text{div} \left( \lambda_{pm} \text{grad} \, T \right) - q^h = 0 \qquad \alpha \in \{w, n\} \end{eqnarray*}

\-This is discretized using a fully-\/coupled vertex centered finite volume (box) scheme as spatial and the implicit \-Euler method as temporal discretization.

\-By using constitutive relations for the capillary pressure $p_c = p_n - p_w$ and relative permeability $k_{r\alpha}$ and taking advantage of the fact that $S_w + S_n = 1$ and $X^\kappa_w + X^\kappa_n = 1$, the number of unknowns can be reduced to two. \-If both phases are present the primary variables are, like in the nonisothermal two-\/phase model, either $p_w$, $S_n$ and temperature or $p_n$, $S_w$ and temperature. \-The formulation which ought to be used can be specified by setting the {\ttfamily \-Formulation} property to either {\ttfamily \-Two\-P\-Two\-Indices\-::p\-Ws\-N} or {\ttfamily \-Two\-P\-Two\-C\-Indices\-::p\-Ns\-W}. \-By default, the model uses $p_w$ and $S_n$. \-In case that only one phase (nonwetting or wetting phase) is present the second primary variable represents a mass fraction. \-The correct assignment of the second primary variable is performed by a phase state dependent primary variable switch. \-The phase state is stored for all nodes of the system. \-The following cases can be distinguished\-:
\begin{itemize}
\item \-Both phases are present\-: \-The saturation is used (either $S_n$ or $S_w$, dependent on the chosen formulation).
\item \-Only wetting phase is present\-: \-The mass fraction of air in the wetting phase $X^a_w$ is used.
\item \-Only non-\/wetting phase is present\-: \-The mass fraction of water in the non-\/wetting phase, $X^w_n$, is used.
\end{itemize}

