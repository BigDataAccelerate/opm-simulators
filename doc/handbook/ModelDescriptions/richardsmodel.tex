%%%%%%%%%%%%%%%%%%%%%%%%%%%%%%%%%%%%%%%%%%%%%%%%%%%%%%%%%%%%%%%%%
% This file has been autogenerated from the LaTeX part of the   %
% doxygen documentation; DO NOT EDIT IT! Change the model's .hh %
% file instead!!                                                %
%%%%%%%%%%%%%%%%%%%%%%%%%%%%%%%%%%%%%%%%%%%%%%%%%%%%%%%%%%%%%%%%%

In the unsaturated zone, Richards' equation is frequently used to approximate the water distribution above the groundwater level. It can be derived from the two-\/phase equations, i.\-e. \[ \frac{\partial\;\phi S_\alpha \rho_\alpha}{\partial t} - \text{div} \left\{ \rho_\alpha \frac{k_{r\alpha}}{\mu_\alpha}\; \mathbf{K}\; \textbf{grad}\left[ p_\alpha - g\rho_\alpha \right] \right\} = q_\alpha, \] where $\alpha \in \{w, n\}$ is the index of the fluid phase, $\rho_\alpha$ is the fluid density, $S_\alpha$ is the fluid saturation, $\phi$ is the porosity of the soil, $k_{r\alpha}$ is the relative permeability for the fluid, $\mu_\alpha$ is the fluid's dynamic viscosity, $\mathbf{K}$ is the intrinsic permeability tensor, $p_\alpha$ is the fluid phase pressure and $g$ is the potential of the gravity field.

In contrast to the \char`\"{}full\char`\"{} two-\/phase model, the Richards model assumes that the non-\/wetting fluid is gas and that it thus exhibits a much lower viscosity than the (liquid) wetting phase. (This assumption is quite realistic in many applications\-: For example, at atmospheric pressure and at room temperature, the viscosity of air is only about $1\%$ of the viscosity of liquid water.) As a consequence, the $\frac{k_{r\alpha}}{\mu_\alpha}$ term typically is much larger for the gas phase than for the wetting phase. Using this reasoning, the Richards model assumes that $\frac{k_{rn}}{\mu_n}$ is infinitely large compared to the same term of the liquid phase. This implies that the pressure of the gas phase is equivalent to the static pressure distribution and that therefore, mass conservation only needs to be considered for the liquid phase.

The model thus choses the absolute pressure of the wetting phase $p_w$ as its only primary variable. The wetting phase saturation is calculated using the inverse of the capillary pressure, i.\-e. \[ S_w = p_c^{-1}(p_n - p_w) \] holds, where $p_n$ is a reference pressure given by the problem's {\ttfamily reference\-Pressure()} method. Nota bene, that the last step assumes that the capillary pressure-\/saturation curve can be uniquely inverted, i.\-e. it is not possible to set the capillary pressure to zero if the Richards model ought to be used!

