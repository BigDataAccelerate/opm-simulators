%%%%%%%%%%%%%%%%%%%%%%%%%%%%%%%%%%%%%%%%%%%%%%%%%%%%%%%%%%%%%%%%%
% This file has been autogenerated from the LaTeX part of the   %
% doxygen documentation; DO NOT EDIT IT! Change the model's .hh %
% file instead!!                                                %
%%%%%%%%%%%%%%%%%%%%%%%%%%%%%%%%%%%%%%%%%%%%%%%%%%%%%%%%%%%%%%%%%

This model solves equations of the form \[ \phi \left( \rho_w \frac{\partial S_w}{\partial t} + \rho_n \frac{\partial S_n}{\partial t}\right) + \textbf{div}\, \boldsymbol{v}_{total} = q. \] The definition of the total velocity $\boldsymbol{v}_{total}$ depends on the choice of the primary pressure variable. Further, fluids can be assumed to be compressible or incompressible (Property\-: {\ttfamily Enable\-Compressibility}). In the incompressible case a wetting $(w) $ phase pressure as primary variable leads to

\[ - \textbf{div}\, \left[\lambda \boldsymbol K \left(\textbf{grad}\, p_w + f_n \textbf{grad}\, p_c + \sum f_\alpha \rho_\alpha \, g \, \textbf{grad}\, z\right)\right] = q, \]

a non-\/wetting ( $ n $) phase pressure yields \[ - \textbf{div}\, \left[\lambda \boldsymbol K \left(\textbf{grad}\, p_n - f_w \textbf{grad}\, p_c + \sum f_\alpha \rho_\alpha \, g \, \textbf{grad}\, z\right)\right] = q, \] and a global pressure leads to \[ - \textbf{div}\, \left[\lambda \boldsymbol K \left(\textbf{grad}\, p_{global} + \sum f_\alpha \rho_\alpha \, g \, \textbf{grad}\, z\right)\right] = q. \] Here, $ p_\alpha $ is a phase pressure, $ p_ {global} $ the global pressure of a classical fractional flow formulation (see e.\-g. P. Binning and M. A. Celia, ''Practical implementation of the fractional flow approach to multi-\/phase flow simulation'', Advances in water resources, vol. 22, no. 5, pp. 461-\/478, 1999.), $ p_c = p_n - p_w $ is the capillary pressure, $ \boldsymbol K $ the absolute permeability, $ \lambda = \lambda_w + \lambda_n $ the total mobility depending on the saturation ( $ \lambda_\alpha = k_{r_\alpha} / \mu_\alpha $), $ f_\alpha = \lambda_\alpha / \lambda $ the fractional flow function of a phase, $ \rho_\alpha $ a phase density, $ g $ the gravity constant and $ q $ the source term.

For all cases, $ p = p_D $ on $ \Gamma_{Dirichlet} $, and $ \boldsymbol v_{total} \cdot \boldsymbol n = q_N $ on $ \Gamma_{Neumann} $.

The slightly compressible case is only implemented for phase pressures! In this case for a wetting $(w) $ phase pressure as primary variable the equations are formulated as \[ \phi \left( \rho_w \frac{\partial S_w}{\partial t} + \rho_n \frac{\partial S_n}{\partial t}\right) - \textbf{div}\, \left[\lambda \boldsymbol{K} \left(\textbf{grad}\, p_w + f_n \, \textbf{grad}\, p_c + \sum f_\alpha \rho_\alpha \, g \, \textbf{grad}\, z\right)\right] = q, \] and for a non-\/wetting ( $ n $) phase pressure as \[ \phi \left( \rho_w \frac{\partial S_w}{\partial t} + \rho_n \frac{\partial S_n}{\partial t}\right) - \textbf{div}\, \left[\lambda \boldsymbol{K} \left(\textbf{grad}\, p_n - f_w \textbf{grad}\, p_c + \sum f_\alpha \rho_\alpha \, g \, \textbf{grad}\, z\right)\right] = q, \] In this slightly compressible case the following definitions are valid\-: $ \lambda = \rho_w \lambda_w + \rho_n \lambda_n $, $ f_\alpha = (\rho_\alpha \lambda_\alpha) / \lambda $ This model assumes that temporal changes in density are very small and thus terms of temporal derivatives are negligible in the pressure equation. Depending on the formulation the terms including time derivatives of saturations are simplified by inserting $ S_w + S_n = 1 $.

In the IMPES models the default setting is: \begin{itemize} 
\item formulation: Property: {\ttfamily Formulation} defined as {\ttfamily DecoupledTwoPCommonIndices::pwSw} 
\item compressibility: disabled Property: {\ttfamily EnableCompressibility} set to {\ttfamily false} 
\end{itemize}



