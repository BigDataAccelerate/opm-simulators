\section{Quick Installation of \eWoms} \label{quick-install}

This only provides one quick way of installing \eWoms.  As a
pre-requisite it is assumed that you are using \eWoms with a recent
Linux distribution that has the appropriate development packages
installed, but without the distribution packages of the \Dune core
modules installed.  If you need more information, or if you have \Dune
already installed, please have a look at the detailed installation
instructions in Section \ref{install}.

\subsection{Retrieving the code}


You can download all \Dune modules by either downloading and unpacking
the tarballs for the \Dune-2.2 release as well as downloading and
unpacking the tarball for the \eWoms 2.2 release, or by retrieving the
code directly from their respective source-code repositories. If you
decide to use the first method, make sure to unpack all tarballs into
the same directory, for the second method you can enter the following
code snipplet into a terminal:
\begin{lstlisting}[style=Bash]
for DUNE_MODULE in common geometry grid istl localfunctions; do \
     git svn clone https://svn.dune-project.org/svn/dune-$DUNE_MODULE/branches/release-2.2 $DUNE_MODULE \
done
git clone --branch "release-2.2" git://github.com/OPM/ewoms.git
\end{lstlisting}

\subsection{Building \Dune and \eWoms}
\label{buildIt}

\eWoms is (almost) a standard \Dune module and is recommended to be
build using the \Dune build system~\cite{DUNE-BS}. \eWoms ships with a
few option files for \Dune's build script, \texttt{dunecontrol}. For
the first time compilation we recommend to use the options optimized
for the debugging experience, \texttt{debug.opts}:

\begin{lstlisting}[style=Bash]
# make sure you are in the DUNE's root directory 
./dune-common/bin/dunecontrol --opts=ewoms/debug.opts all
\end{lstlisting}

If you finished with developing and testing your own code on
small-scale problems, re-compile everything with compiler
optimizations enabled before a production run:

\begin{lstlisting}[style=Bash]
# make sure you are in the DUNE's root directory
./dune-common/bin/dunecontrol --opts=ewoms/optim.opts all
\end{lstlisting}

Sometimes it is necessary to have additional options which are
specific to the concrete operating system you use or you might also
have special needs.  For this reason, the option files mentioned above
are to be rather understood as a starting point for specifying the own
options than as something which is fixed; feel free to copy and modify
them.  For example, if you need external libraries, you might need to
add or modify quite many options.  To avoid confusion, it can also be
helpful to use a different name for your customized option files.

%%% Local Variables: 
%%% mode: latex
%%% TeX-master: "ewoms-handbook"
%%% End: 
