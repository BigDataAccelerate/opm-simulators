\section{Installation of \Dumux} \label{install}
\subsection{Preliminary remarks}

In this section about the installation of \Dumux it is assumed that you work on a UNIX compatible operating system and that you are familiar with the use of a shell. 
Moreover, you should know,  if necessary, how to install new software packages or you should have a person aside which can give you assistance with the shell and the package installation. 
Below, we list some basic prerequisites for running \Dune and \Dumux. 
Please check this paragraph to evaluate whether it is possible for you to run \Dune and \Dumux. 
Moreover, some optional libraries and modules are listed, which can be helpful for the work with \Dumux. \\

In a technical sense \Dumux is a module of \Dune. 
The installation procedure of \Dumux is strongly related to that of \Dune. 
Thus, for details of the installation please check also the {\Dune} website \cite{DUNE-HP}. 
If you are interested in more details of the build-system we refer to the {\Dune}-Build-System-Howto \cite{DUNE-HP}.\\

As in a \Dune installation, all \Dune modules including \Dumux should be extracted into a common directory. We refer to that directory for purpose of documentation abstactly as {\Dune}-ROOT direcory or shortly as {\Dune}-ROOT, but in real life the user is free to choose any valid directory name for his instance of {\Dune}-ROOT direcory.

% subject to change
Each \Dune module is associated with a directory name in the folder {\Dune}-ROOT. For convenience, this directory name is
also used as an alias for the module name, a practice we will follow here, too.

%But this convention is sometimes broken, e.g.
%when you do the build out of tar-files. The associated directory name is not identical with module name but it is modulename
%enhanced by a version suffix, e.g. version-number something like \texttt{dumux-2.0} which is also a valid directory name for modle  \Dune root for module \texttt{dumux} in version 2.0. 
%To refer abstractly the directory name where the files of module named modulename are in we also use later on the term MODULENAME-ROOT, e.g. DUMUX-ROOT means just path to directory of module \texttt{dumux}.


After installing source code for all required \Dune modules including the \Dumux parts, \Dune is being built by the shell-command \texttt{dunecontrol} which is part of the {\Dune}-Build-System. The {\Dune}-Build-System is essentially a front-end of the gnu tool \texttt{autoconf} which was specialized for the \Dune project.

\paragraph{Basic prerequisites} \label{prerequisites}
The gnu tool-chain of \texttt{g++}  and related gnu variants of autotools (ubuntu: autoconf, automake, libtool, build-essential) must be available in a recent version. 
The \Dumux property system, which is used in most of the models, makes use of \texttt{libboost}. 
It is thus necessary to install a  \texttt{boost}  version $\geqslant 1.33.1$ (ubuntu: libboost-dev). 
The building of the documentation requires \LaTeX\ and auxiliary tools like \texttt{dvipdf} and \texttt{bibtex}. 
If you use the configuration switch \texttt{--enable-doxygen} in order to generate the doxygen files (documentation generator) you will also need \texttt{doxygen}.

Depending on the usage of external libraries the required software packages may vary. 
Beside the section on external modules below it is a good idea to check the {\Dune}-Wiki \cite{DUNE-HP} for information about the external packages.

\subsection{Obtaining source code for \Dune and \Dumux}
From version 2.0 of \Dune on, it was decided to stick to stable \Dune releases, comprising the core modules 
\texttt{dune-common}, \texttt{dune-grid}, \texttt{dune-istl}, \texttt{dune-localfunctions} and the external dune module \texttt{dune-pdelab}. Thus for a complete \Dumux installation these modules are required to be installed. Of course the external dune module \texttt{dumux} from \Dumux website ist required too. Using further modules can be necessary, if you are going to use more than basic features of \Dune or \Dumux. To some of these we only refer briefly in end of this chapter.\\

Two possibilities exist to get source code for \Dune and \Dumux.
Firstly \Dune and \Dumux can obtained as tar-archive files by download from respective {\Dune} and {\Dumux} website. They then need to be extracted, as described in the next section.
Secondly, described in the section next but one, is a method to obtain most recent source or even also every predecessors by direct access via internet to software repositories of software revision control system of the software developer. \Dune and \Dumux use for their software repositories \texttt{apache subversion} or more shortly \texttt{subversion} or \texttt{svn}.
As a user does not always want the most recent version
certain version tags (i.e. special names) and version numbers and even software branches are means of software revision control system to provide different versions of a software from a software repository.

\paragraph{Obtaining the software by installing tar-files}
The a bit old fashioned named tape-archive-file shortly named tar-file or tarball is a common file format for distributing collections of files contained in these archives.
This part of installation is done as follows: 
Download the tarballs from the respective \Dune (version 2.0) and \Dumux websites to a certain path in your filesystem.
Create the {\Dune}-ROOT directory, named here DUMUX, then extract content of tar-files by commandline program tar into it.
Above, except download, it can achieved by the following shell commands:
\begin{itemize}
\item \texttt{mkdir DUMUX}
\item \texttt{cd DUMUX}
\item \texttt{tar xzvf path\_to\_tarball\_of/dune-common-2.0.tar.gz }
\item \texttt{tar xzvf path\_to\_tarball\_of/dune-grid-2.0.tar.gz }
\item \texttt{tar xzvf path\_to\_tarball\_of/dune-istl-2.0.tar.gz }
\item \texttt{tar xzvf path\_to\_tarball\_of/dune-localfunctions-2.0.tar.gz }
\item \texttt{tar xzvf path\_to\_tarball\_of/dumux-2.0.tar.gz}
\end{itemize} 
Optional:
\begin{itemize}
\item \texttt{tar xzvf path\_to\_tarball\_of/dune-grid-howto-2.0.tar.gz}
\end{itemize} 
However, the required \Dune-module \texttt{dune-pdelab} is not available as tar-file, thus you have to install from software repository by the second method.  Thus if svn command is available, it can done as follows: 
\begin{itemize}
\item \texttt{svn checkout https://svn.dune-project.org/svn/dune-pdelab/branches/2.0snapshot dune-pdelab}
\end{itemize} 

\paragraph{Obtaining \Dune and \Dumux from software repositories} 

Using a software revision control system as subversion and its software repositories has beside it advantages for developer or developing groups as a tool for assisting them to keep the source code worked on in consistent fashion for an user, i.e. a member who is not in the developing group of a software project, of software also some advantages.
In general With direct access to the software repositories it is easier to keep up with code changes, to receive important bug fixes and to keep up with general developments of code.

To access apache subversion software repositories a certain program is needed which is referred here shortly as \texttt{subversion client} or \texttt{svn client}. We use in our description the svn client of the apache subversion software itself, which is a commandline (shell) tool, mostly named \texttt{svn}. 
Apache subversion client is available for most Linux and UNIX distributions as software package by the distributor or can get installed by building it directly from source.

In technical speach of apache subversion ``checking out a certain software" version means nothing more then fetching 
it from software repository and laying it out in the filesystem, additionally with software some files with metainformation regarding the software revision control system itself are also laid out.
For a developer in \Dumux project it is easily possible to do the opposite i.e. uploading a modified version he created into software repository. This is usually termed as ``software check in".

The installation procedure is as follows:
Create {\Dune}-ROOT directory, named here below DUMUX.
Then, enter the previously created directory and checkout the modules. 
As you see below the checkout uses two different servers for sources one for \Dune and one for \Dumux, also the 
anonymous access to these repositories varies a little bit. 

The checkout for \Dune is as described on \Dune's website, \cite{DUNE-HP}.
\begin{itemize}
\item \texttt{mkdir DUMUX}
\item \texttt{cd DUMUX}
\item \texttt{svn checkout https://svn.dune-project.org/svn/dune-common/releases/2.0 dune-common}
\item \texttt{svn checkout https://svn.dune-project.org/svn/dune-grid/releases/2.0 dune-grid}
\item \texttt{svn checkout https://svn.dune-project.org/svn/dune-istl/releases/2.0 dune-istl}
\item \texttt{svn checkout https://svn.dune-project.org/svn/dune-localfunctions/releases/2.0 dune-localfunctions}
\item \texttt{svn checkout https://svn.dune-project.org/svn/dune-pdelab/branches/2.0snapshot dune-pdelab}
\end{itemize} 

The \texttt{dune-grid} HOWTO is a tutorial which aims to give an understanding of the \Dune grid interface, which can 
give you an idea how abstractions in \Dune are done. \texttt{Dune-grid} HOWTO is not required by \Dumux, thus installing is optional. It is done by: 
\begin{itemize}
\item \texttt{svn checkout \\
      \hspace{4cm} https://svn.dune-project.org/svn/dune-grid-howto/releases/2.0 dune-grid-howto}
\end{itemize}

%\paragraph{Checkout of \Dumux}

To obtain the needed \texttt{dumux} module, which is as mentioned in the above technical an external module of \Dune. You have to check it out as also described on \Dumux website. %, \cite{DUMUX-HP}
Its filetree should laid out as for any other \Dune module directly into \Dune-ROOTs directory. This is done if you are in it and execute the following command within \Dune-ROOT.

\begin{itemize}
\item \texttt{svn checkout --username=anonymous --password=''\\
    \hspace{4cm}  svn://svn.iws.uni-stuttgart.de/DUMUX/dumux/trunk dumux}
\end{itemize}

\paragraph{Hints for \Dumux-Developers}
If you also want to check in your own \Dumux developments to \Dumux software repositories, you can apply for either full developers access or access for developer access on certain parts of \Dumux. Granted developer access means that you are allowed to check in own code to the software repository and that you can access the \texttt{dumux-devel} module which enhances \texttt{dumux} by staging code of developer group, which is unavailable for user not being member of the developer group.
If you are member of developer group rights the checkout with developer access for \texttt{dumux} and \texttt{dumux-dev} is non-anonymously done with the commands below, for insert your username
for accessing \Dumux software repository in place \texttt{yourusername} below.
\begin{itemize}
 \item \texttt{svn checkout --username=yourusername \\ 
      \hspace{4cm} svn://svn.iws.uni-stuttgart.de/DUMUX/dumux/trunk dumux}
 \item \texttt{svn checkout --username=yourusername \\
      \hspace{4cm} svn://svn.iws.uni-stuttgart.de/DUMUX/dune-mux/trunk dumux-devel}
\end{itemize} 
\texttt{Dumux-devel} itself makes use of stable part \texttt{dumux} and hence it needs also being checked out. 
You can omit in commands above the username option, if your username for the repository access is identical to the one for your system account.
Please choose either not to store the password or to store it by subversion in a secure way (check the documentation of subversion for that). 
A leaked out password can be used by evil persons to vandalize a software repository.

\paragraph{External libraries and modules}

The following libraries provide additional functionality but are not required to run \Dumux. 
If you are going to use an external library check the information provided on the \Dune website.

\begin{itemize}
\item \textbf{UG}: UG is a toolbox for Unstructured Grids: As \Dumux, it is build by \texttt{autotools} and a C++-compiler. Additionally, the tools \texttt{lex}/\texttt{yacc} or the gnu-versions \texttt{flex}/\texttt{bison} are needed. 

\item \textbf{Alberta}: Adaptive multi Level finite element toolbox using Bisectioning refinement and Error control by Residual Techniques for scientific Applications: A Fortran compiler like \texttt{gfortran} is required.

\item \textbf{ALUGrid}: ALUGrid is build by a C++-compiler like \texttt{g++}. If you want to build a parallel version, you will need \texttt{MPI}. It was successfully run with \texttt{openmpi}. The parallel version needs also a graph partitioner, such as \texttt{METIS} or \texttt{PARTY}. It was run successfully in combination with \Dune using \texttt{METIS}.

\item \textbf{PARDISO}: The package PARDISO is a thread-safe, high-performance, robust, memory efficient and easy to use software for solving large sparse symmetric and asymmetric linear systems of equations on shared memory multiprocessors. The precompiled binary can be downloaded after personal registration from the PARDISO website (\texttt{\url{http://www.pardiso-project.org}}).

\item \textbf{SuperLU}: SuperLU is a general purpose library for the direct solution of large, sparse, nonsymmetric systems of linear equations (\texttt{\url{http://crd.lbl.gov/~xiaoye/SuperLU}}).

\item \textbf{\Dune-multidomaingrid}: If you going to run on the same grid different domains or subdomains, this can be the package of choice. 
To be precise, this is not only an external library but an external \Dune module which, like \Dumux, is built by the \Dune build system. The \Dune-multidomaingrid module provides a meta grid that allows the subdivison of arbitrary \Dune grids into subdomains, e.g. for multi-physics approaches (\texttt{\url{http://gitorious.org/dune-multidomaingrid}}).
\end{itemize}

The following are dependencies of some of the used libraries. You will need them depending on which modules of \Dune and which external libraries you use.
\begin{itemize}
\item \textbf{MPI}: The parallel version of \Dune and also some of the external dependencies need MPI when they are going to be built for parallel computing. \texttt{Openmpi} and \texttt{MPICH} in a recent version have been reported to work. 

\item \textbf{lex/yacc} or \textbf{flex/bison}: These are quite common developing tools, code generators for lexical analyzers and parsers. This is a prerequisite for UG.

\item \textbf{BLAS}: Alberta makes use of BLAS. Thus install libGOTO, ATLAS, non-optimized BLAS or BLAS by chipmanufacturer. Take care that the installation scripts select the intended version of BLAS. 

\item \textbf{METIS}: This is a dependency of ALUGrid. If you run it parallel this part is being used to partition your grid.

\item \textbf{libGOTO}: is an optimized version of BLAS. It is not always available for all architectures and 
the license is not open. For research and education it is possible to obtain a copy without additional costs.
A Fortran compiler like \texttt{gfortran} is needed to compile it.

\item \textbf{\LaTeX, doxygen}: In order to build the \Dumux documentation these tools are needed.  
\LaTeX\ is a front-end to the \TeX\ font setting system. Install texlive for example. A great part of the code documentation is done inline using doxygen, which extracts the documentation and makes HTML or \LaTeX\ out of that.

\item \textbf{Compilers}: We recommend to use the gnu compiler suite, i.e. \texttt{gcc}/\texttt{g++}/\texttt{gfortran} version 4.4 or 4.5.
It is also reported that \Dune was successfully build with the Intel C++ compiler. 
\Dune is build by a C++ compiler. The C and Fortran compiler is only needed for a few external libraries.

\item \textbf{libgomp}: \Dune can make use of OpenMP. Thus it can be necessary to install the \texttt{libgomp} library.
% http://openmp.org/

\item \textbf{libgmp}: The Gnu Multiple Precision Arithmetic Library (GMP) is also a prerequisite for \Dune. It may be necessary to install it.
% http://gmplib.org/
\end{itemize}

\paragraph{Hint for IWS Users} Users from the Institute of Hydraulic Engineering, University of Stuttgart,
can also checkout a svn repository with several external libraries: \\

\texttt{svn checkout svn://svn.iws.uni-stuttgart.de/DUMUX/external/trunk external}\\

This directory \texttt{external} contains a script to install external libraries, such as 
\texttt{alberta}, \texttt{alu}, \texttt{ug}. It also contains METIS and the BLAS version of libGOTO2: 
\begin{center}
\texttt{./installExternal.sh all}
\end{center}
it is also possible to install only certain parts of the external libraries:
\begin{center}
\texttt{./installExternal.sh alberta}
\end{center}

The libraries are then build within the external-directory and are not installed in a different place. 
But a \Dune build still needs to know where they are. You have to refer to them as options for the \Dune build, for example in your options file \texttt{debug.opts}.


\paragraph{Build \Dune and \Dumux}
\label{buildIt}
To compile \Dumux without additional options file, type in the {\Dune}-ROOT-directory (called \texttt{DUMUX} before):
\begin{center}
 \texttt{./dune-common/bin/dunecontrol --configure-opts="CXXFLAGS=-fno-strict-aliasing"  all}
\end{center}

If you want to compile in the debugging mode, type in the folder \texttt{DUMUX}: 
\begin{center}
\texttt{./dune-common/bin/dunecontrol --opts=\$DUMUX\_ROOT/debug.opts all}
\end{center}

Otherwise you can type
\begin{center}
\texttt{./dune-common/bin/dunecontrol --opts=\$DUMUX\_ROOT/optim.opts all}
\end{center}
in order to get an optimized compilation (better performance, but no possibility to use a debugger). The expression \verb+\$DUMUX\_ROOT+
needs to be replaced by the path to the directory \texttt{dumux-devel} was checked out to. 

This uses the \Dune build-system. If it does not work, please have a look at the file \texttt{INSTALL} in the \Dumux root directory (if you use SVN, this \texttt{\$DUMUX\_ROOT} is usually \texttt{dumux}, 
if you use a released version it is usually \texttt{dumux-VERSION}). 
You can also find more information in the \Dune Buildsystem HOWTO located at the \Dune web page, \cite{DUNE-HP}.  
Alternatively, the tool CMake can be used to build \Dumux. Please check the file \texttt{INSTALL.cmake} for details.
