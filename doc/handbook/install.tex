\section{Installation} 

For the installation of DuMu$^\text{x}$, the following steps have to be performed.  

\paragraph{Checkout of the core modules}
Since we always want to be up to date with the latest changes in DUNE, 
we decided to use the developers version of the core modules 
\texttt{dune-common}, \texttt{dune-grid}, \texttt{dune-istl}, \texttt{dune-disc}, 
and \texttt{dune-grid-howto}. First, create a directory where all the DUNE modules will be stored in. Then, enter the previously created folder. The checkout has to be performed as described on 
the DUNE webpage, \cite{DUNE-HP}: 
\begin{itemize}
\item \texttt{svn checkout https://svn.dune-project.org/svn/dune-common/trunk dune-common}
\item \texttt{svn checkout https://svn.dune-project.org/svn/dune-grid/trunk dune-grid}
\item \texttt{svn checkout https://svn.dune-project.org/svn/dune-istl/trunk dune-istl}
\item \texttt{svn checkout https://svn.dune-project.org/svn/dune-disc/trunk dune-disc}
\item \texttt{svn checkout https://svn.dune-project.org/svn/dune-grid-howto/trunk dune-grid-howto}
\end{itemize} 

\paragraph{Checkout of DuMu$^\text{x}$ and external modules} 

First of all, you need to ask one of the IWS system administrators to 
add your account to the group \texttt{svndune}. 
If you are working on a LH2 computer, you then can checkout DuMu$^\text{x}$ 
and the external modules via 
\begin{itemize}
\item \texttt{svn checkout svn+ssh://luftig/home/svn/DUMUX/dune-mux/trunk dune-mux}
\item \texttt{svn checkout svn+ssh://luftig/home/svn/DUMUX/dune-subgrid/trunk dune-subgrid}
\item \texttt{svn checkout svn+ssh://luftig/home/svn/DUMUX/external/trunk external}
\end{itemize} 
If you want to checkout from outside LH2, you first need to establish a tunnel. 
To this end, you need to add once 
\begin{center}
\texttt{ssht = ssh -p 2022 -l login -o HostKeyAlias=luftig.iws.uni-stuttgart.de} 
\end{center}
to the file \texttt{\$HOME/.subversion/config}, with \texttt{login} replaced 
by your actual IWS login name. 
The tunnel then needs to be initialized every time you want 
to connect to the repository by 
\begin{center}
\texttt{ssh -Nf -L 2022:luftig.iws.uni-stuttgart.de:22 login@login1.iws.uni-stuttgart.de}.
\end{center}
Then, you can checkout everything as described above, if you replace \texttt{luftig} 
by \texttt{localhost}. 

\paragraph{Build the external modules} 
The external modules consist of Alberta, ALUGrid, UG, GotoBLAS, and Paraview. 
To install them all, execute the script \texttt{installXXbit.sh} in the folder \texttt{external}. 
If you like to only install some of the external software, you can copy the corresponding 
parts of the script and paste them to the command line. 
Please also refer to the DUNE webpage for additional details, \cite{DUNE-HP}. 

\paragraph{Build DUNE and DuMu$^\text{x}$}
Type in the folder \texttt{DUMUX}: 
\begin{center}
\texttt{./dune-common/bin/dunecontrol --opts=dune-mux/debug.opts all}
\end{center}

