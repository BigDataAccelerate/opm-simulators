\chapter{The \Dumux Fluid Framework}
\label{sec:fluidframework}

This chapter discusses the \Dumux fluid framework. \Dumux users who
do not want to write new models and who do not need new fluid
configurations may skip this chapter.

In the chapter, a high level overview over the the principle concepts
is provided first, then some implementation details follow.

\section{Overview of the Fluid Framework}

The \Dumux fluid framework currently features the following concepts
(listed roughly in their order of importance):

\begin{description}
\item[Fluid state:] Fluid states are responsible for representing the
  complete thermodynamic configuration of a system at a given spatial
  and temporal position. A fluid state always provides access methods
  to {\bf all} thermodynamic quantities, but the concept of a fluid state does not
  mandate what assumptions are made to store these thermodynamic
  quantities. What fluid states also do {\bf not} do is to make sure
  that the thermodynamic state which they represent is physically
  possible.
\item[Fluid system:] Fluid systems express the thermodynamic {\bf
    relations}\footnote{Strictly speaking, these relations are
    functions, mathematically.} between quantities. Since functions do
  not exhibit any internal state, fluid systems are stateless classes,
  i.e. all member functions are \texttt{static}. This is a conscious
  decision since the thermodynamic state of the system is expressed by
  a fluid state!
\item[Parameter cache:] Fluid systems sometimes require
  computationally expensive parameters for multiple relations. Such
  parameters can be cached using a so-called parameter
  cache. Parameter cache objects are specific for each fluid system
  but they must provide a common interface to update the internal
  parameters depending on the quantities which changed since the last
  update.
\item[Constraint solver:] Constraint solvers are auxiliary tools to
  make sure that a fluid state is consistent with some thermodynamic
  constraints. All constraint solvers specify a well defined set of
  input variables and make sure that the resulting fluid state is
  consistent with a given set of thermodynamic equations. See section
  \ref{sec:constraint_solvers} for a detailed description of the
  constraint solvers which are currently available in \Dumux.
\item[Equation of state:] Equations of state (EOS) are auxiliary
  classes which provide relations between a fluid phase's temperature,
  pressure, composition and density. Since these classes are only used
  internally in fluid systems, their programming interface is
  currently ad-hoc.
\item[Component:] Components are fluid systems which provide the
  thermodynamic relations for the liquid and gas phase of a single
  chemical species or a fixed mixture of species. Their main purpose
  is to provide a convenient way to access these quantities from
  full-fledged fluid systems. Components are not supposed to be used
  by models directly.
\item[Binary coefficient:] Binary coefficients describe the relations
  of a mixture of two components. Typical binary coefficients are
  \textsc{Henry} coefficients or binary molecular diffusion
  coefficients. So far, the programming interface for accessing binary
  coefficients has not been standardized in \Dumux.
\end{description}

\section{Fluid States}

Fluid state objects express the complete thermodynamic state of a
system at a given spatial and temporal position. 

\subsection{Exported Constants}

{\bf All} fluid states {\bf must} export the following constants:
\begin{description}
\item[numPhases:] The number of fluid phases considered.
\item[numComponents:] The number of considered chemical
  species or pseudo-species.
\end{description}

\subsection{Accessible Thermodynamic Quantities}

Also, {\bf all} fluid states {\bf must} provide the following methods:
\begin{description}
\item[temperature():] The absolute temperature $T_\alpha$ of
  a fluid phase $\alpha$.
\item[pressure():] The absolute pressure $p_\alpha$ of a
  fluid phase $\alpha$.
\item[saturation():] The saturation $S_\alpha$ of a fluid phase
  $\alpha$. The saturation is defined as the pore space occupied by
  the fluid divided by the total pore space:
  \[
  \saturation_\alpha := \frac{\porosity \mathcal{V}_\alpha}{\porosity \mathcal{V}}
  \]
\item[moleFraction():] Returns the molar fraction $x^\kappa_\alpha$ of
  the component $\kappa$ in fluid phase $\alpha$. The molar fraction
  $x^\kappa_\alpha$ is defined as the ratio of the number of molecules
  of component $\kappa$ and the total number of molecules of the phase
  $\alpha$.
\item[moleFraction():] Returns the mass fraction $X^\kappa_\alpha$ of
  component $\kappa$ in fluid phase $\alpha$. The mass fraction
  $X^\kappa_\alpha$ is defined as the weight of all molecules of a
  component divided by the total mass of the fluid phase. It is
  related with the component's mole fraction by means of the relation
  \[
  X^\kappa_\alpha = x^\kappa_\alpha \frac{M^\kappa}{\overline M_\alpha}\;,
  \]
  where $M^\kappa$ is the molar mass of component $\kappa$ and
  $\overline M_\alpha$ is the mean molar mass of a molecule of phase
  $\alpha$.
\item[averageMolarMass():] Returns $\overline M_\alpha$, the mean
  molar mass of a molecule of phase $\alpha$. For a mixture of $N > 0$
  components, $\overline M_\alpha$ is defined as
  \[
  \overline M_\alpha = \sum_{\kappa=1}^{N} x^\kappa_\alpha M^\kappa
  \]
\item[density():] Returns the density $\rho_\alpha$ of the fluid phase
  $\alpha$.
\item[molarDensity():] Returns the molar density $\rho_{mol,\alpha}$
  of a fluid phase $\alpha$. The molar density is defined by the mass
  density $\rho_\alpha$ and the mean molar mass $\overline M_\alpha$:
  \[
  \rho_{mol,\alpha} = \frac{\rho_\alpha}{\overline M_\alpha} \;.
  \]
\item[molarVolume():] Returns the molar volume $v_{mol,\alpha}$ of a
  fluid phase $\alpha$. This quantity is the inverse of the molar
  density.
\item[molarity():] Returns the molar concentration $c^\kappa_\alpha$
  of component $\kappa$ in fluid phase $\alpha$.
\item[fugacity():] Returns the fugacity $f^\kappa_\alpha$ of component
  $\kappa$ in fluid phase $\alpha$. The fugacity is defined as
  \[
  f_\alpha^\kappa := \Phi^\kappa_\alpha x^\kappa_\alpha p_\alpha \;,
  \]
  where $\Phi^\kappa_\alpha$ is the {\em fugacity
    coefficient}~\cite{reid1987}.  The physical meaning of fugacity
  becomes clear from the equation
  \[
  f_\alpha^\kappa = p_\alpha \exp\left\{\frac{\zeta^\kappa_\alpha}{R T_\alpha} \right\} \;,
  \]
  where $\zeta^\kappa_\alpha$ represents the $\kappa$'s chemical
  potential in phase $\alpha$, $R$ stands for the ideal gas constant,
  and $T_\alpha$ for the absolute temperature of phase
  $\alpha$. Assuming thermal equilibrium, there is a one-to-one
  mapping between a component's chemical potential
  $\zeta^\kappa_\alpha$ and its fugacity $f^\kappa_\alpha$. In this
  case chemical equilibrium can thus be expressed by
  \[
  f^\kappa := f^\kappa_\alpha = f^\kappa_\beta\quad\forall \alpha, \beta
  \]
\item[fugacityCoefficient():] Returns the fugacity coefficient
  $\Phi^\kappa_\alpha$ of component $\kappa$ in fluid phase $\alpha$.
\item[enthalpy():] Returns specific enthalpy $h_\alpha$ of a fluid
  phase $\alpha$.
\item[internalEnergy():] Returns specific internal energy $u_\alpha$
  of a fluid phase $\alpha$. The specific internal energy is defined
  by the relation
  \[
  u_\alpha = h_\alpha - \frac{p_\alpha}{\rho_\alpha}
  \]
\item[viscosity():] Returns the dynamic viscosity
  $\mu_\alpha$ of fluid phase $\alpha$.
\end{description}
  
\subsection{Available Fluid States}
Currently, the following fluid states are provided by \Dumux:
\begin{description}
\item[NonEquilibriumFluidState:] This is the most general fluid state
  supplied. It does not assume thermodynamic equilibrium and thus
  stores all phase compositions (using mole fractions), fugacity
  coefficients, phase temperatures, phase pressures, saturations and
  specific enthalpies.
\item[CompositionalFluidState:] This fluid state is very similar to
  the \texttt{Non\-Equilibrium\-Fluid\-State} with the difference that
  the \texttt{Compositional\-Fluid\-State} assumes thermodynamic
  equilibrium. In the context of multi-phase flow in porous media,
  this means that only a single temperature needs to be stored.
\item[ImmisicibleFluidState:] This fluid state assumes that the fluid
  phases are immiscible, which implies that the phase compositions and
  the fugacity coefficients do not need to be stored explicitly.
\item[PressureOverlayFluidState:] This is a so-called {\em overlay}
  fluid state. It allows to set the pressure of all fluid phases but
  forwards everything else to another fluid state.
\item[SaturationOverlayFluidState:] This fluid state is like the
  \texttt{PressureOverlayFluidState}, except that the phase
  saturations are settable instead of the phase pressures.
\item[TempeatureOverlayFluidState:] This fluid state is like the
  \texttt{PressureOverlayFluidState}, except that the temperature is
  settable instead of the phase pressures. Note that this overlay
  state assumes thermal equilibrium regardless of underlying fluid
  state.
\item[CompositionOverlayFluidState:] This fluid state is like the
  \texttt{PressureOverlayFluidState}, except that the phase
  composition is settable (in terms of mole fractions) instead of the
  phase pressures.
\end{description}

\section{Fluid Systems}

Fluid systems express the thermodynamic relations between the
quantities of a fluid state.

\subsection{Parameter Caches}

All fluid systems must export a type for their \texttt{ParameterCache}
objects. Parameter caches can be used to cache parameter that are
expensive to compute and are required in multiple thermodynamic
relations. For fluid systems which do need to cache parameters,
\Dumux provides a \texttt{NullParameterCache} class.

The actual quantities stored by parameter cache objects are specific
to the fluid system and no assumptions on what they provide should be
made outside of their fluid system. Parameter cache objects provide a
well-defined set of methods to make them coherent with a given fluid
state, though. These update are:
\begin{description}
\item[updateAll(fluidState, except):] Update all cached quantities for
  all phases. The \texttt{except} argument contains a bit field of the
  quantities which have not been modified since the last call to a
  \texttt{update()} method.
\item[updateAllPresures(fluidState):] Update all cached quantities
  which depend on the pressure of any fluid phase.
\item[updateAllTemperatures(fluidState):] Update all cached quantities
  which depend on temperature of any fluid phase.
\item[updatePhase(fluidState, phaseIdx, except):] Update all cached
  quantities for a given phase. The quantities specified by the
  \texttt{except} bit field have not been modified since the last call
  to an \texttt{update()} method.
\item[updateTemperature(fluidState, phaseIdx):] Update all cached
  quantities which depend on the temperature of a given phase.
\item[updatePressure(fluidState, phaseIdx):] Update all cached
  quantities which depend on the pressure of a given phase.
\item[updateComposition(fluidState, phaseIdx):] Update all cached
  quantities which depend on the composition of a given phase.
\item[updateSingleMoleFraction(fluidState, phaseIdx, compIdx):] Update
  all cached quantities which depend on the value of the mole fraction
  of a component in a phase.
\end{description}
Note, that the parameter cache interface only guarantees that if a
more specialized \texttt{update()} method is called, it is not slower
than the next more-general method (e.g. calling
\texttt{updateSingleMoleFraction()} may be as expensive as
\texttt{updateAll()}). It is thus advisable to rather use a more
general \texttt{update()} method once than multiple calls to
specialized \texttt{update()} methods.

To make usage of parameter caches easier for the case where all cached
quantities ought to be re-calculated if a quantity of a phase was
changed, it is possible to only define the \texttt{updatePhase()}
method and derive the parameter cache from
\texttt{Dumux::ParameterCacheBase}.

\subsection{Exported Constants and Capabilities}

Besides providing the type of their \texttt{ParameterCache} objects,
fluid systems need to export the following constants and auxiliary
methods:
\begin{description}
\item[numPhases:] The number of considered fluid phases.
\item[numComponents:] The number of considered chemical (pseudo-)
  species.
\item[init():] Initialize the fluid system. This is usually used to
  tabulate some quantities
\item[phaseName():] Given the index of a fluid phase, return its name
  as human-readable string.
\item[componentName():] Given the index of a component, return its
  name as human-readable string.
\item[isLiquid():] Return whether the phase is a liquid, given the
  index of a phase.
\item[isIdealMixture():] Return whether the phase is an ideal mixture,
  given the phase index. In the context of the \Dumux fluid
  framework a phase $\alpha$ is an ideal mixture if, and only if, all
  its fugacity coefficients $\Phi^\kappa_\alpha$ do not depend on the
  phase composition. (Although they might very well depend on
  temperature and pressure.)
\item[isIdealGas():] Return whether a phase $\alpha$ is an ideal gas,
  i.e. it adheres to the relation
  \[
  p_\alpha v_{mol,\alpha} = R T_\alpha \;,
  \]
  with $R$ being the ideal gas constant.
\item[isCompressible():] Return whether a phase $\alpha$ is
  compressible, i.e. its density depends on pressure $p_\alpha$.
\item[molarMass():] Given a component index, return the molar mass of
  the corresponding component.
\end{description}

\subsection{Thermodynamic Relations}

Fluid systems have been explicitly designed to provide as few
thermodynamic relations as possible. A full-fledged fluid system thus
only needs to provide the following thermodynamic relations:
\begin{description}
\item[density():] Given a fluid state, an up-to-date parameter cache
  and a phase index, return the mass density $\rho_\alpha$ of the
  phase.
\item[fugacityCoefficient():] Given a fluid state, an up-to-date
  parameter cache as well as a phase and a component index, return the
  fugacity coefficient $\Phi^\kappa_\alpha$ of a the component for the
  phase.
\item[viscosity():] Given a fluid state, an up-to-date parameter cache
  and a phase index, return the dynamic viscosity $\mu_\alpha$ of the
  phase.
\item[diffusionCoefficient():] Given a fluid state, an up-to-date
  parameter cache, a phase and a component index, return the calculate
  the molecular diffusion coefficient for the component in the fluid
  phase.
  
  Molecular diffusion of a component $\kappa$ in phase $\alpha$ is
  caused by a gradient of the chemical potential and follows the law
  \[ 
  J^\kappa_\alpha = - D^\kappa_\alpha\ \mathbf{grad} \zeta^\kappa_\alpha\;,
  \] 
  where $\zeta^\kappa_\alpha$ is the component's chemical potential,
  $D^\kappa_\alpha$ is the diffusion coefficient and $J^\kappa_\alpha$
  is the diffusive flux. $\zeta^\kappa_\alpha$ is connected to the
  component's fugacity $f^\kappa_\alpha$ by the relation
  \[ 
  \zeta^\kappa_\alpha = 
  R T_\alpha \mathrm{ln} \frac{f^\kappa_\alpha}{p_\alpha} \;.
  \]
\item[binaryDiffusionCoefficient():] Given a fluid state, an
  up-to-date parameter cache, a phase index and two component indices,
  return the binary diffusion coefficient for the binary mixture. This
  method is less general than \texttt{diffusionCoefficient} method,
  but relations can only be found for binary diffusion coefficients in
  the literature.
\item[enthalpy():] Given a fluid state, an up-to-date parameter cache
  and a phase index, this method calulates the specific enthalpy
  $h_\alpha$ of the phase.
\item[thermalConductivity:] Given a fluid state, an up-to-date
  parameter cache and a phase index, this method returns the thermal
  conductivity $\lambda_\alpha$ of the fluid phase. The thermal
  conductivity is defined by means of the relation
  \[
  \dot Q = \lambda_\alpha \mathbf{grad}\;T_\alpha \;,
  \]
  where $\dot Q$ is the heat flux caused by the temperature gradient
  $\mathbf{grad}\;T_\alpha$.
\item[heatCapacity():] Given a fluid state, an up-to-date parameter
  cache and a phase index, this method computes the isobaric heat
  capacity $c_{p,\alpha}$ of the fluid phase. The isobaric heat
  capacity is defined as the partial derivative of the specific
  enthalpy $h_\alpha$ to the fluid pressure:
  \[
  c_{p,\alpha} = \frac{\partial h_\alpha}{\partial p_\alpha}
  \]
  % TODO: remove the heatCapacity() method??
\end{description}

Fluid systems may chose not to implement some of these methods and
throw an exception of type \lstinline{Dune::NotImplemented} instead. Obviously,
such fluid systems cannot be used for models that depend on those
methods.

\subsection{Available Fluid Systems}

Currently, the following fluid systems are available in \Dumux:
\begin{description}
\item[Dumux::FluidSystems::TwoPImmiscible:] A two-phase fluid system
  which assumes immiscibility of the fluid phases. The fluid phases
  are thus completely specified by means of their constituting
  components. This fluid system is intended to be used with models
  that assume immiscibility.
\item[Dumux::FluidSystems::H2ON2:] A two-phase fluid system featuring
  gas and liquid phases and distilled water ($H_2O$) and pure
  molecular Nitrogen ($N_2$) as components.
\item[Dumux::FluidSystems::H2OAir:] A two-phase fluid system
  featuring gas and liquid phases and distilled water ($H_2O$) and
  air (Pseudo component composed of $79\%\;N_2$, $20\%\;O_2$ and
  $1\%\;Ar$) as components.
\item[Dumux::FluidSystems::H2OAirMesitylene:] A three-phase fluid
  system featuring gas, NAPL and water phases and distilled water, air
  and Mesitylene ($C_6H_3(CH_3)_3$) as components. This fluid system
  assumes all phases to be ideal mixtures.
\item[Dumux::FluidSystems::H2OAirXylene:] A three-phase fluid system
  featuring gas, NAPL and water as phases and distilled water, air and
  Xylene ($C_8H_{10}$) as components. This fluid system assumes all
  phases to be ideal mixtures.
\item[Dumux::FluidSystems::Spe5:] A three-phase fluid system featuring
  gas, oil and water as phases and the seven components distilled
  water, Methane ($C_1$), Propane ($C_3$), Pentane ($C_5$), Heptane
  ($C_7$), Decane ($C_{10}$), Pentadecane ($C_{15}$) and Icosane
  ($C_{20}$). For the water phase the IAPWS-97 formulation is used as
  equation of state, while for the gas and oil phases a
  \textsc{Peng}-\textsc{Robinson} equation of state with slightly
  modified parameters is used. This fluid system is highly non-linear,
  and the gas and oil phases also cannot be considered ideal
  mixtures\cite{SPE5}.
\end{description}

\section{Constraint Solvers}
\label{sec:constraint_solvers}

Constraint solvers connect the thermodynamic relations expressed by
fluid systems with the thermodynamic quantities stored by fluid
states. Using them is not mandatory for models, but given the fact
that some thermodynamic constraints can be quite complex to solve,
sharing this code between models makes sense. Currently, \Dumux
provides the following constraint solvers:
\begin{description}
\item[CompositionFromFugacities:] This constraint solver takes all
  component fugacities, the temperature and pressure of a phase as
  input and calculates the composition of the fluid phase. This means
  that the thermodynamic constraints used by this solver are
  \[
  f^\kappa = \Phi^\kappa_\alpha(\{x^\beta_\alpha \}, T_\alpha, p_\alpha)  p_\alpha x^\kappa_\alpha\;,
  \]
  where ${f^\kappa}$, $T_\alpha$ and $p_\alpha$ are fixed values.
\item[ComputeFromReferencePhase:] This solver brings all
  fluid phases into thermodynamic equilibrium with a reference phase
  $\beta$, assuming that all phase temperatures and saturations have
  already been set. The constraints used by this solver are thus
  \begin{eqnarray*}
  f^\kappa_\beta = f^\kappa_\alpha = \Phi^\kappa_\alpha(\{x^\beta_\alpha \}, T_\alpha, p_\alpha)  p_\alpha x^\kappa_\alpha\;, \\
  p_\alpha = p_\beta + p_{c\beta\alpha} \;,
  \end{eqnarray*}
  where $p_{c\beta\alpha}$ is the capillary pressure between the
  fluid phases $\beta$ and $\alpha$.
\item[NcpFlash:] This is a so-called flash solver. A flash solver
  takes the total mass of all components per volume unit and the phase
  temperatures as input and calculates all phase pressures,
  saturations and compositions. This flash solver works for an
  arbitrary number of phases $M > 0$ and components $N \geq M - 1$. In
  this case, the unknown quantities are the following:
  \begin{itemize}
  \item $M$ pressures $p_\alpha$
  \item $M$ saturations $\saturation_\alpha$
  \item $M\cdot N$ mole fractions $x^\kappa_\alpha$
  \end{itemize}
  This sums up to $M\cdot(N + 2)$. The equations side of things
  provides:
  \begin{itemize}
  \item $(M - 1)\cdot N$ equations stemming from the fact that the
    fugacity of any component is the same in all phases, i.e.
    \[
    f^\kappa_\alpha = f^\kappa_\beta
    \]
    holds for all phases $\alpha, \beta$ and all components $\kappa$.
  \item $1$ equation comes from the fact that the whole pore space is
    filled by some fluid, i.e.
    \[
    \sum_{\alpha=1}^M \saturation_\alpha = 1
    \]
  \item $M - 1$ constraints are given by the capillary pressures:
    \[ 
    p_\beta = p_\alpha + p_{c\beta\alpha} \;,
    \]
    for all phases $\alpha$, $\beta$
  \item $N$ constraints come the fact that the total mass of each
    component is given:
    \[
    c^\kappa_{tot} = \sum_{\alpha=1}^M x_\alpha^\kappa\;\rho_{mol,\alpha} = const
    \]
  \item And finally $M$ model assumptions are used. This solver uses
    the NCP constraints proposed in~\cite{LHHW2011}:
    \[
     0 = \mathrm{min}\{\saturation_\alpha, 1 - \sum_{\kappa=1}^N x_\alpha^\kappa\}
   \]
\end{itemize}
The number of equations also sums up to $M\cdot(N + 2)$. Thus, the
system of equations is closed.
\item[ImmiscibleFlash:] This is a flash solver assuming immiscibility
  of the phases. It is similar to the \texttt{NcpFlash} solver but a
  lot simpler.
\item[MiscibleMultiphaseComposition:] This solver calculates the
  composition of all phases provided that each of the phases is
  potentially present. Currently, this solver does not support
  non-ideal mixtures.
\end{description}

%%% Local Variables: 
%%% mode: latex
%%% TeX-master: "dumux-handbook"
%%% End: 
