\section{Guidelines} 

This section mainly quotes the DUNE coding guidelines found at \cite{DUNE-HP}.
"In order to keep the code maintainable we have decided upon a set of coding rules. 
Some of them may seem like splitting hairs to you, but they do make it much easier 
for everybody to work on code that hasn't been written by oneself.

\begin{itemize}
\item Naming: 
\begin{itemize}
\item Variables: Names for variables should only consist of letters and digits. The first letter should be a lower case one. If your variable names consists of several words, then the first letter of each new word should be capital. As we decided on the only exception are the begin and end methods.
\item Private Data Variables: Names of private data variables end with an underscore.
\item Typenames: For typenames, the same rules as for variables apply. The only difference is that the first letter should be a capital one.
\item Macros: The use of preprocessor macros is strongly discouraged. If you have to use them for whatever reason, please use capital letters only.
\item The Exlusive-Access Macro: Every header file traditionally begins with the definition of a preprocessor constant that is used to make sure that each header file is only included once. If your header file is called 'myheaderfile.hh', this constant should be DUNE\_MYHEADERFILE\_HH.
\item Files: Filenames should consist of lower case letters exclusively. Header files get the suffix .hh, implementation files the suffix .cc
\end{itemize}
\item Documentation:
      Dune, as any software project of similar complexity, will stand and fall with the quality of its documentation.
Therefore it is of paramount importance that you document well everything you do! We use the doxygen system to extract easily-readable documentation from the source code. Please use its syntax everywhere. In particular, please comment all
\begin{itemize}
\item Method Parameters
\item Template Parameters
\item Return Values
\item Exceptions thrown by a method
 \end{itemize}
     Since we all know that writing documentation is not well-liked and is frequently defered to some vague 
'next week', we herewith proclaim the Doc-Me Dogma . It goes like this: Whatever you do, and in whatever hurry you 
happen to be, please document everything at least with a {\verb /** $\backslash$todo Please doc me! */}. That way at least the absence 
of documentation is documented, and it is easier to get rid of it systematically.
\item Exceptions:
      The use of exceptions for error handling is encouraged. Until further notice, all exceptions thrown are DuneEx.
\item Debugging Code:
      Global debugging code is switched off by setting the symbol NDEBUG. In particular, all asserts are 
automatically removed. Use those asserts freely!" 
\end{itemize}
